% uncomment this to compile separately
% \DocumentMetadata{
%     lang=en-US,
%     tagging=on,
%     pdfversion=2.0,
%     pdfstandard=ua-2,
%     tagging-setup={math/setup=mathml-SE},
%     colorprofiles={
%         A = sRGB.icc, %or longer GTS_PDFA1 = sRGB.icc
%         X = sRGB.icc,
%         ISO_PDFE1 = sRGB.icc
%     },
%     % pdftitle={Lecture notes - accessibility test},
%     % pdfauthor={Silas Mitchell}
% }

\documentclass{article}
\usepackage{notes}

\begin{document}

\setcounter{section}{1}

\section{Chapter 2}

\subsection{Equations of lines}

Whenever we have two points, we can draw a line through them. When we do this on the coordinate plane, we can describe the line with an equation.

We'll first deal with lines that are also functions (so, not vertical lines).

\subsubsection{Properties of lines}

\begin{definition}{(Intercepts)}{def:intercepts}
    \begin{itemize}
        \item An $x$-intercept is the coordinate where a line crosses the $x$ axis.\\We can find the $x$-intercept from an equation by setting $y=0$ and solving for $x$.
        \item A $y$-intercept is the coordinate where a line crosses the $y$ axis.\\We can find the $y$-intercept from an equation by setting $x=0$ and solving for $y$.
    \end{itemize}
\end{definition}
Not all lines have both of these; for example, a horizontal line might not have an $x$-intercept, and a vertical line might not have a $y$-intercept.

Another important property of lines is the direction in which they point. We can think of this as how steeply the line increases/decreases.

\begin{definition}{(Slope)}{def:slope}
    The slope of a line tells us its rate of change. Larger numbers mean steeper lines, and negative numbers mean the line slants downwards.

    We can calculate slope from two points using the formula \[m = \frac{y_2-y_1}{x_2-x_1}=\frac{\Delta y}{\Delta x}=\frac{\text{rise}}{\text{run}}.\]
\end{definition}

\begin{prop}{}{prop:line_fct_y_intercept}
    All lines which are functions have a $y$-intercept and a slope.
\end{prop}

\begin{example}{}{ex:finding_slope}
    Find the slope of a line through the points:
    \begin{problem}
        \item $(0,0)$ and $(2,6)$
        \item $(-3,5)$ and $(4,7)$
        \item $(2,5)$ and $(3,2)$
    \end{problem}
\end{example}
\begin{solution}
    We use the slope formula $m=\frac{y_2-y_1}{x_2-x_1}$.
    \begin{problem}
        \item Plugging into the slope formula, we get \[m=\frac{6-0}{2-0}=\frac{6}{2}=3.\] 
        \item \[m=\frac{7-5}{4-(-3)}=\frac{2}{7}.\]
        \item \[m=\frac{2-5}{3-2}=\frac{-3}{1}=-3.\]
    \end{problem}
\end{solution}

\subsubsection{Equations of lines}

Given the slope and $y$-intercept of a line, we can write an equation for the line.
\begin{definition}{(Slope-intercept form)}{def:slope_intercept}
    The line with slope $m$ and $y$-intercept $b$ can be represented by the equation $y=mx+b$.
\end{definition}

\begin{example}{}{ex:slope_intercept}
    Find the equation in slope-intercept form for the line with slope $3$ and $y$-intercept $-5$.
\end{example}
\begin{solution}
    We know $m=3$ and $b=-5$. Plugging this into the formula $y=mx+b$, we get \[y=3x-5.\]
\end{solution}

\begin{example}{(Textbook 2.1 example 6)}{ex:2.1.6}
    The distance $y$ in miles that an athlete training for a marathon is from home after $x$ hours is shown below.
    
    \begin{minipage}{.35\textwidth}
        \begin{tikzpicture}[scale=1.75,alt={graph of athlete's distance from home}]
            \draw[very thick,->] (0,0) -- (2.2,0) node[anchor=west] {$x$};
            \draw[very thick,->] (0,0) -- (0,2.2) node[anchor=south] {$y$};
            \draw[thin,gray,step=.5] (0,0) grid (2.1,2.1);
            \draw[blue] (0,1.5) -- (1.5,0);
            \foreach \x in {0,.5,...,2} {
                \draw (\x,0) node[anchor=north] {$\x$};
            }
            \foreach \y in {5,10,...,20} {
                \draw (0,\y/10) node[anchor=east] {$\y$};
            }
            \filldraw (0,1.5) circle (1pt);
            \filldraw (1,.5) circle (1pt) node[anchor=south west] {$(1,5)$};
            \filldraw (1.5,0) circle (1pt);
        \end{tikzpicture}
    \end{minipage}
    \begin{minipage}{.63\textwidth}
        \begin{problem}
            \item Find the $y$-intercept. Interpret your answer.
            \item Find the $x$-intercept. Interpret your answer.
            \item The graph passes through the point $(1,5)$. Discuss the meaning of this point.
            \item Find the slope of this line. Interpret the slope as a rate of change.
            \item Write the slope-intercept form of this line.
        \end{problem}
    \end{minipage}
\end{example}
\begin{solution}
    \begin{problem}
        \item The line goes through the point $(0,15)$, which lies on the $y$-axis. Therefore, the $y$-intercept is 15. This means that the athlete starts 15 miles from home.
        \item The line goes through the point $(1.5,0)$ on the $x$-axis, so the $x$-intercept is $1.5$. This means that it takes the athlete 1.5 hours to reach home.
        \item After 1 hour, the athlete is 5 miles from home.
        \item To find the slope, we need two points. Let's use $(0,15)$ and $(1.5,0)$. Then we use the slope formula: \[m=\frac{0-15}{1.5-0}=\frac{-15}{1.5}=-10.\] In the context of the problem, this means the athlete gets 10 miles closer to home every hour, or that he is running home at 10 mph.
        Note that we could do this with any two points on the line, and we would get the same answer.
        \item We know from part (a) that $b=15$, and from part (d) that $m=-10$. Thus we get $y=-10x+15$ for the slope-intercept form of the line.
    \end{problem}
\end{solution}
% \includegraphics[width=6.5in]{Textbook_Mandatory_Examples/Exam1/2.1/2.1.6}

Sometimes we know a point other than the $y$-intercept on the line. In this case, we can use point-slope form to write an equation.
\begin{definition}{(Point-slope form)}{def:point_slope}
    The line with slope $m$ passing through point $(x_1,y_1)$ has the equation \[y-y_1=m(x-x_1)\quad\text{or}\quad y=m(x-x_1)+y_1\] in point-slope form.
\end{definition}

\begin{example}{(Textbook 2.1 ex 1)}{ex:2.1.1}
    Find an equation of the line passing through the points $(-2,-3)$ and $(1,3)$. Plot the points and graph the line by hand.
\end{example}
\begin{solution}
    To find an equation of the line, we need to know the slope. We find the slope using the two points we're given and the slope formula: \[m=\frac{3-(-3)}{1-(-2)}=\frac{6}{3}=2.\] Now we choose one of the points and use that point and the slope to find its point-slope form.
    If we use the first point, we get $y=2(x-(-2))+(-3)$, and if we use the second we get $y=2(x-1)+3$. These are both valid answers.

    Now we want to graph the line:
    \begin{center}\begin{tikzpicture}[scale=.4
        ,alt={graph of the given line}]
        \draw[very thick,<->] (-3.2,0) -- (3.2,0) node[anchor=west] {$x$};
        \draw[very thick,<->] (0,-3.2) -- (0,3.2) node[anchor=north east] {$y$};
        \draw[gray,thin] (-3.1,-3.1) grid (3.1,3.1);
        \draw[blue,<->] (-2.1,-3.2) -- (1.1,3.2);
        \filldraw (-2,-3) circle (2pt);
        \filldraw (1,3) circle (2pt);
    \end{tikzpicture}\end{center}
\end{solution}
% \includegraphics[width=6.5in]{Textbook_Mandatory_Examples/Exam1/2.1/2.1.1.png}

From the point-slope form, we can simplify to get the equation in slope-intercept form.

\begin{example}{(Textbook 2.1 ex 2)}{ex:2.1.2}
    Find the point-slope form for the line that satisfies the conditions. Then convert this equation into slope-intercept form and write the formula for a function $f$ whose graph is the line.
    \begin{problem}
        \item Slope $-\frac{1}{2}$, passing through the point $(-3,-7)$
        \item $x$-intercept $-4$, $y$-intercept $2$
    \end{problem}
\end{example}
\begin{solution}
    \begin{problem}
        \item We have $m=-\frac{1}{2}$ and $(x_1,y_1)=(-3,-7)$. Plugging these into the formula gives us $y-(-7)=-\frac{1}{2}(x-(-3))$. Then we simplify:
        \begin{align*}
            y+7&=-\frac{1}{2}(x+3)\\
            &=-\frac{1}{2}x-\frac{3}{2}\\
            y&=-\frac{1}{2}x-\frac{3}{2}-7\\
            &=-\frac{1}{2}x-\frac{17}{2}.
        \end{align*}
        \item We have two points: $(-4,0)$ and $(0,2)$. We need to find the slope: $m=\frac{2-0}{0-(-4)}=\frac{2}{4}=\frac{1}{2}$. Now we can plug into the point-slope formula using either point. Let's use the first; that gives us $y-0=\frac{1}{2}(x-(-4))$. Now we simplify:
        \begin{align*}
            y-0&=\frac{1}{2}(x+4)\\
            y&=\frac{1}{2}x+2.
        \end{align*}
        We would get the same thing if we started by using the point-slope form given by the other point and simplifying, too.
    \end{problem}
\end{solution}
% \includegraphics[width=6.5in]{Textbook_Mandatory_Examples/Exam1/2.1/2.1.2.png}

There is one more form that we might see, called standard form; this is not part of the content for this class, but is included for completeness and because it has applications in linear algebra.
\begin{definition}{(Standard form equation of a line)}{def:line_standard_form}
    An equation of a line is in standard form when it is written as $ax+by=c$ for constants $a,b,c$ where $a$ and $b$ are not both 0. We can write the equation for any line in this form, even vertical lines which are not functions.
\end{definition}

\subsubsection{Horizontal and vertical lines}

A horizontal line is a function, so is covered by the cases we've already seen. Horizontal lines are constant functions, and have slope 0.
\begin{definition}{(Equation of a horizontal line)}{def:horizontal_line_eqn}
    A horizontal line is given by the formula $f(x)=b$ for some constant $b$ (which is the $y$-intercept).
\end{definition}

A vertical line is not a function.
\begin{definition}{(Equation of a vertical line)}{def:vertical_line_eqn}
    A vertical line is given by the equation $x=k$, where $k$ is the $x$-intercept.
\end{definition}

\begin{example}{(Textbook 2.1 ex 7)}{ex:2.1.7}
    Find equations of vertical and horizontal lines passing through the point $(8,5)$. If possible, for each line write a formula for a linear function whose graph is the line.
\end{example}
\begin{solution}
    First, consider the vertical line. We know that vertical lines have the form $x=k$, and that we must pass through a point with $x$ coordinate $8$. Therefore, this line is described by the equation $x=8$.

    Now consider the horizontal line. We know that horizontal lines have the form $y=b$, and this line passes through a point with $y$ coordinate $5$, so the line is $y=5$. Alternatively, we could use the fact that horizontal lines have slope 0 to write this line in point-slope form, and then simplify. This line is a function; as a function, we would write it as $f(x)=5$.

    \begin{center}\begin{tikzpicture}[alt={Graph of the desired lines}]
        \draw[very thick,<->] (-1.1,0) -- (4.6,0) node[anchor=west] {$x$};
        \draw[very thick,<->] (0,-1.1) -- (0,3.1) node[anchor=south] {$y$};
        \filldraw (4,2.5) node[anchor=north east] {(8,5)} circle (1pt);
        \draw[blue,<->] (-1,2.5) -- (4.5,2.5);
        \draw[red,<->] (4,-1) -- (4,3);
    \end{tikzpicture}
    \end{center}
\end{solution}
% \includegraphics[width=6.5in]{Textbook_Mandatory_Examples/Exam1/2.1/2.1.7.png}

\subsubsection{Parallel and perpendicular lines}

\begin{definition}{(Parallel lines)}{def:parallel}
    Two lines are parallel if they never cross.
    \begin{itemize}
        \item Any vertical lines are parallel to each other
        \item Two nonvertical lines are parallel if and only if their slopes are equal
    \end{itemize}
\end{definition}

\begin{example}{(Textbook 2.1 ex 8)}{ex:2.1.8}
    Find the slope-intercept form of a line parallel to $y=-2x+5$, passing through $(4,3)$
\end{example}
\begin{solution}
    For two non-vertical lines to be parallel, they must have the same slope. The line we are given is in slope-intercept form, so we can easily see that its slope is $m=-2$.

    We know that our line will have slope $-2$ and goes through the point $(4,3)$. Using this information, we can write the line in point-slope form, then simplify to get its slope-intercept form:
    \begin{align*}
        y-3&=-2(x-4)\\
        y&=-2(x-4)+3\\
        &=-2x+8+3\\
        &=-2x+11.
    \end{align*}
    Therefore, the line with the desired properties is given by $y=-2x+11$.
\end{solution}
% \includegraphics[width=6.5in]{Textbook_Mandatory_Examples/Exam1/2.1/2.1.8.png}

\begin{definition}{(Perpendicular lines)}{def:perpendicular}
    Two lines are perpendicular if they cross at a right angle.
    \begin{itemize}
        \item Any horizontal line is perpendicular to any vertical line
        \item Two nonvertical lines are perpendicular if their slopes multiply to $-1$ (or, if the slope of one is the {\it negative reciprocal} of the slope of the other)
    \end{itemize}
\end{definition}

\begin{example}{(Textbook 2.1 ex 9)}{ex:2.1.9}
    Find the slope-intercept form of the line perpendicular to $y=-\frac{2}{3}x+2$, passing through the point $(-2,1)$. Graph the lines.
\end{example}
\begin{solution}
    The line we're given has slope $-\frac{2}{3}$. When two lines are perpendicular, their slopes multiply to $-1$, so we have $-\frac{2}{3}\cdot m=-1$; dividing both sides by $-\frac{2}{3}$ gives us $m=\frac{3}{2}$. Now we can use the slope and the given point to write the line in point-slope form, then simplify to get slope-intercept form:
    \begin{minipage}{.49\textwidth}
        \begin{align*}
            y-1&=\frac{3}{2}(x-(-2))\\
            y&=\frac{3}{2}(x+2)+1\\
            &=\frac{3}{2}x+3+1\\
            &=\frac{3}{2}x+4.
        \end{align*}
    \end{minipage}
    \begin{minipage}{.49\textwidth}
        \begin{center}\begin{tikzpicture}[scale=.7,alt={graph of the two perpendicular lines}]
            \draw[very thick,<->] (-1.2,0) -- (4.2,0) node[anchor=west] {$x$};
            \draw[very thick,<->] (0,-1.2) -- (0,4.2) node[anchor=south] {$y$};
            \draw[gray,thin] (-1.1,-1.1) grid (4.1,4.1);
            \draw[blue,<->] plot[domain=-1.2:4.2] (\x,-2/3*\x+2);
            \draw[red,<->] plot[domain=-1.2:1.5] (\x,3/2*\x+2);
        \end{tikzpicture}\end{center}
    \end{minipage}
\end{solution}
% \includegraphics[width=6.5in]{Textbook_Mandatory_Examples/Exam1/2.1/2.1.9.png}

\subsubsection{Interpolation and extrapolation}

We can use linear equations to make estimations based on known data points.
\begin{definition}{(Interpolation and extrapolation)}{def:interpolation_extrapolation}
    \begin{itemize}
        \item Estimating using a line segment between two data points is called interpolation
        \item Estimating outside of known data by extending a line is called extrapolation
    \end{itemize}
\end{definition}

\includegraphics[width=6.5in]{Textbook_Mandatory_Examples/Exam1/2.1/2.1.11.png}


% Point slope
% slope intercept
% from a graph
% horizontal/vertical
% parallel and perpendicular lines
% interpolation/extrapolation


\end{document}